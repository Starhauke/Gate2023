% \iffalse
\let\negmedspace\undefined
\let\negthickspace\undefined
\documentclass[journal,12pt,twocolumn]{IEEEtran}
\usepackage{cite}
\usepackage{amsmath,amssymb,amsfonts,amsthm}
\usepackage{algorithmic}
\usepackage{graphicx}
\usepackage{textcomp}
\usepackage{xcolor}
\usepackage{txfonts}
\usepackage{listings}
\usepackage{enumitem}
\usepackage{mathtools}
\usepackage{gensymb}
\usepackage{comment}
\usepackage[breaklinks=true]{hyperref}
\usepackage{tkz-euclide} 
\usepackage{listings}
\usepackage{gvv}  
\usepackage{tikz}
\usepackage{circuitikz} 
\usepackage{caption}

\def\inputGnumericTable{}                                
\usepackage[latin1]{inputenc}                 
\usepackage{color}                            
\usepackage{array}                            
\usepackage{longtable}                        
\usepackage{calc}                            
\usepackage{multirow}                      
\usepackage{hhline}                           
\usepackage{ifthen}                          
\usepackage{lscape}
\usepackage{amsmath}
\newtheorem{theorem}{Theorem}[section]
\newtheorem{problem}{Problem}
\newtheorem{proposition}{Proposition}[section]
\newtheorem{lemma}{Lemma}[section]
\newtheorem{corollary}[theorem]{Corollary}
\newtheorem{example}{Example}[section]
\newtheorem{definition}[problem]{Definition}
\newcommand{\BEQA}{\begin{eqnarray}}
\newcommand{\EEQA}{\end{eqnarray}}
\newcommand{\define}{\stackrel{\triangle}{=}}
\theoremstyle{remark}
\newtheorem{rem}{Remark}

\begin{document}
\title{}
\author{Sasa Mardi, EE23BTECH11222}
\date{}
\maketitle

\textbf{Question NM 25:} Consider the contour integral $\oint \frac{dz}{z^4 + z^3 - 2z^2}$, along the curve $|z| = 3$ oriented in the counterclockwise direction. If $\text{Res}[f, z_0]$ denotes the residue of $f(z)$ at the point $z_0$, then which of the following are TRUE? \\
\begin{itemize}
    \item (A) $\text{Res}[f, 0] = -\frac{1}{4}$
    \item (B) $\text{Res}[f, 1] = \frac{1}{3}$
    \item (C) $\text{Res}[f, -2] = -\frac{1}{12}$
    \item (D) $\text{Res}[f, 2] = -1$
\end{itemize}
\hfill{(GATE NM 2023)}\\
\solution
\begin{align}
\frac{dz}{z^4 + z^3 - 2z^2} = \frac{dz}{z^2(z-1)(z+2)}
\end{align}
Poles: $z = 0, 1, -2$ \\
Curve: $|z| = 3$, all poles inside it \\
Given the function $f(z) = \frac{1}{z^2(z-1)(z+2)}$, with poles at $z = 0$, $z = 1$, and $z = -2$, and considering the curve $|z| = 3$, where all poles are inside it. We want to find the residues of $f(z)$ at these poles.

The general formula for finding the residue at a pole $z_0$ is:
\begin{align}
\text{Res}(f, z_0) &= \frac{1}{(n-1)!} \lim_{z \to z_0} \frac{d^{n-1}}{dz^{n-1}} \left( (z - z_0)^n f(z) \right)
\end{align}
where $n$ denotes how many times the pole is repeated.

For $z_0 = 0$, where $n = 2$, we have:
\begin{align}
\text{Res}(f, 0) &= \frac{1}{1!} \lim_{z \to 0} \frac{d}{dz} \left( \frac{1}{(z-1)(z+2)} \right) \\
&= -\frac{1}{4}
\end{align}

For $z_0 = 1$, where $n = 1$, we have:
\begin{align}
\text{Res}(f, 1) &= \frac{1}{(1-1)!} \lim_{z \to 1} \frac{z-1}{z^2(z+2)} \\
&= \frac{1}{3}
\end{align}

For $z_0 = -2$, where $n = 1$, we have:
\begin{align}
\text{Res}(f, -2) &= \frac{1}{(1-1)!} \lim_{z \to -2} \frac{z-(-2)}{z^2(z-1)} \\
&= -\frac{1}{12}
\end{align}\\
\text{Therefore, the correct answers are: option A, B and C}
\end{document}
